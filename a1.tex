\documentclass{article}
\usepackage{amsmath,amssymb}
\usepackage{hyperref}
\title{CSC236 Winter 2015, Assignment 1}
\author{Anish Krishna, Connor Peet, George Wu}
\renewcommand{\today}{~}
\hypersetup{pdfpagemode=Fullscreen,
  colorlinks=true,
  linkfileprefix={}}
\newcommand{\floor}[1]{\lfloor #1\rfloor}
\begin{document}
\maketitle

\begin{enumerate}

\item Proving $1 + mn \leq (1 + m)^n$.
	\begin{description}
	\item Let $P(n, m)$ be that $1 + mn \leq (1 + m)^n$.
	\item Base Case: $P(0, 0)$
		\begin{description}
		\item $1 + mn \leq (1 + m)^n$
		\item $1 + 0 \times 0 \leq (1 + 0)^0$
		\item $1 \leq 1$ is true
		\end{description}
	\item Inductive Step for $P(n + 1, m)$: $\forall n, m \in \mathbb{N}, P(n, m) \implies P(n + 1, m)$
		\begin{description}
		\item Assume $n, m \in \mathbb{N}$
			\begin{description}
			\item Assume $P(n, m)$ \# Inductive Hypothesis
				\begin{description}
				\item Then $1 + mn \leq (1 + m)^n$ \# By Inductive Hypothesis
				\item Then $1 + mn \leq (1 + m)^n + m \times (1 + m)^n - m$ \# since $(1 + m)^n \geq 1$
				\item Then $1 + mn +  m\leq (1 + m)^n + (1 + m)^n$
				\item Then $1 + mn + m \leq (1 + m)(1 + m)^n$
				\item Then $1 + m(n + 1) \leq (1 + m)^{n + 1}$
				\item Then $P(n + 1, m)$
				\end{description}
			\end{description}
		\end{description}
	\item Inductive Step for $P(n, m + 1)$: $\forall n, m \in \mathbb{N}, P(n, m) \implies P(n, m + 1)$
		\begin{description}
		\item Assume $n, m \in \mathbb{N}$
			\begin{description}
			\item Assume $P(n, m)$ \# Inductive Hypothesis
				\begin{description}
				\item Then $1 + mn \leq (1 + m)^n$ \# By Inductive Hypothesis
				\item Then $1 + mn \leq (1 + m)^n + 2^n - n$ \# $2^n \geq n$ since $n \in  \mathbb{N}$
				\item By binomial theorem, $(1 + m)^n = 1^0m^n + 1^1m^{n-1} \cdots 1^nm^0$ and $(2 + m)^n = 2^0m^n + 2^1m^{n-1} \cdots 2^nm^0$
				\item Then $1 + mn \leq (2 + m)^n - n$ \# by binomial theorem
				\item Then $1 + mn + n \leq (2 + m)^2$
				\item Then $1 + n(m + 1) \leq (1 + (m + 1))^2$
				\item Then $P(n, m + 1)$
				\end{description}
			\end{description}
		\end{description}
	\item Conclude $\forall n, m \in \mathbb{N}, P(n, m)$
	\end{description}
\item Proving 3. a) is:
\begin{equation}
b_{h+1} = b_{h}\sum_{i=0}^{h-1}{b_i} + b_{h}\sum_{i=0}^{h-2}{b_i}
\end {equation}
    \begin{description}
			\item By the definition of a binary tree, a tree of height $h$ as subtrees of height $h-1$ and height $<h$
			\item Then the total number of trees of height $h-1$ (case for the left subtree) times the sum of all possible trees of height $<h$ (case for the right subtree) added to the total number of trees of height $h-1$(case for the right subtree) times the sum of all possible trees of height $<h-1$ (case for the left subtree).
			\item In other words:
			\begin{equation}
			b_{h+1} = b_{h}\sum_{i=0}^{h-1}{b_i} + b_{h}\sum_{i=0}^{h-2}{b_i}
			\end {equation}
    \end{description}
\item Proving 3. b) is:
		\begin{description}
			\item By Complete Induction.
			\item \textbf{Base Case $a_0$, $b_0$:}
			\item $a_0$ = 0 \#By definition
			\item $b_0$ = 1 \#By definition
			\item \textbf{Inductive step}
			\item Assume
			\item $\forall k \in N, k< h, P(k)$
			\item Observing
			\begin {equation}
			b_{h+1} = b_{h} \sum_{i = 0}^{h}b_i + b_h \sum_{i = 0}^{h-1}b_i
			\end {equation}
			\item \# by 3. a)
			\item Then $b_h\times [b_0 +b_1 + ... + b_h + b_0 + b_1 + ... + b_{h-1}]$
			\item Then $b_h \times [1 + a_1^2 - a_0^2 + a_2^2-a_1^2 + ... + a_h^2 -a_{h-1}^2 + 1 + a_1^2 - a_0^2 + ... + a_{h-1}^2 - a_{h-2}^2]$ \# By Inductive hypothesis
			\item $b_h \times [a_h^2 + 1 + a_{h-1}^2 + 1]$ \# algebraic manipulation of a telescopic series.
			\item $(a_h^2 - a_{h-1}^2) \times [a_{h+1} + a_h]$ \# definition of $b_h$, and by definition of $a_{h+1} = a_h^2 + 1$
			\item $(a_h^2+1 - a_{h-1}^2-1) \times [a_{h+1} + a_h]$ \# Adding "zero", by adding one and subtracting one.
			\item $(a_{h+1} - a_{h}) \times [a_{h+1} + a_h]$ \# definition of $a_{h+1}$, and through algebraic manipulation.
			\item Thus, $b_{h+1} = (a_{h+1}^2 - a_{h}^2)$
		\end{description}

\end{enumerate}

\end{document}

% LocalWords:  brachial
